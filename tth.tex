\chapter{Search for \ttHbb}

In this chapter, we will describe the search for~\ttHbb\xspace using the matrix element method at the CMS experiment (MEM) in Run II of the LHC. This is based on preliminary results from CMS~\cite{CMS:2016qwm,CMS:2016zbb} and ongoing work. We concentrate on the semileptonic (SL) and dileptonic (DL) decay channels of the top quark pair and the application of the MEM in this search, where various the primary background arises from QCD production of \ttbar+jets. We are showing results with the 2016 dataset of CMS data, the use of which for this thesis has been endorsed by the CMS Higgs group.

\begin{figure}
\begin{centering}
\subfloat{\includegraphics[width=0.3\textwidth]{figures/tth/feyn_tth.png}}
\subfloat{\includegraphics[width=0.3\textwidth]{figures/tth/feyn_tth1.png}} \\
\caption[Representative leading-order Feynman diagrams for the \ttHbb\xspace process]{Representative leading-order Feynman diagrams for the \ttHbb\xspace process in the gluon-fusion production channel. On the left, we show the dileptonic (DL) decay of the top quark pair, whereas on the left, we show the semileptonic (SL) decay channel.}
\label{fig:tth_diagrams}
\end{centering}
\end{figure}

Briefly, the analysis proceeds as follows. Throughout the analysis, we use the MC simulation and data samples described in~\cref{sec:data_mc}. The detailed identification criteria for the physics objects (jets and leptons) are motivated by established top quark analyses and are described in~\cref{sec:object_id}. The selection is motivated by the semileptonic and dileptonic decays of the top quark pair, which results in a final state containing several jets, charged leptons and \MET. We select events with at least 1 (2) charged lepton(s) in the semileptonic (dileptonic) channel and at least 4 jets, out of which 3 must be b-tagged.

The statistical analysis is described in~\cref{sec:analysis}. We further divide the events into independent categories based on the jet and b-tag multiplicity, as described in~\cref{sec:event_selection}. These categories are introduced in order to constrain the various sub-processes of~\ttbar+jets, described further in~\cref{sec:ttbar_subprocesses} together with the \ttH signal cross-section. In order to extract the signal strength modifier~$\mu = \sigma/\sigma_{\mathrm{SM}}$ for the \ttH~process, we perform a combined template fit in all the categories, relying on the discriminating power provided by the MEM in categories with a high signal-to-background ratio and on other multivariate techniques in background-enhanced control regions. The fit is described in~\cref{sec:statistical_method}.

A crucial component in the fit is the estimated systematic uncertainty, which drives the determination of the confidence interval of the estimated signal strength and is described in~\cref{sec:systematic_unc}. We study the fit model with respect to the systematic uncertainties in~\cref{sec:model_analysis} before determining the compatibility of the model with data. Finally, we discuss the results of the analysis in~\cref{sec:tth_results} and present a summary and outlook in~\cref{sec:tth_summary}.

\section{Data and simulation}
\label{sec:data_mc}

We use proton-proton collision data collected by the CMS experiment at a center-of-mass energy of~$\sqrt{s} = 13~\mathrm{TeV}$, corresponding to a total integrated luminosity of~$35.9~\ifb$~in the semileptonic (SL) and dileptonic (DL) channels. We have used the complete dataset of 2016 as an update over the analysis finalized in the middle of 2016, which used roughly $12.9~\ifb$ of data~\cite{CMS:2016zbb}.

We use MC simulation to model the signal and background processes, interfaced to a parton shower and hadronization as appropriate. In order to model the detector effects, we use a detailed simulation of the reconstruction, selection efficiencies and detector resolutions based on \geant. For the~\ttH~signal, \ttbar+jets and single-top backgrounds, we use the NLO generator \powheg~\cite{Frixione:2007vw,Re:2010bp}. The use of an NLO model for the modelling of signal and the primary background is a significant advancement over Run I analyses, where only a LO model was available. Besides~\ttbar+jets, we need to model the production of W or Z/$\gamma^*$~bosons with additional jets (denoted W/Z+jets or commonly V+jets), which is simulated using \madgraphatnlo (v. 2.2.2)~\cite{Hirschi:2011pa} and diboson production (WW, WZ, ZZ), simulated using \pythia~\cite{Sjostrand:2007gs}. Throughout, we assume the value of the top quark mass to be~$m_t \simeq 172.5\GeV/c^2$~and of the Higgs boson to be~$m_H \simeq 125\GeV/c^2$. In order to describe the substructure of the protons via the parton density functions, we use the PDF parametrization provided by NNPDF3.0 and~\pythia~for showering and hadronization.

In order to model the production of hadrons, the parameters in the~\pythia~model have been tuned to historical Tevatron, LEP and LHC data~\cite{CMS-PAS-GEN-14-001,Skands:2014pea}. It has been observed that the default tune in~\pythia~does not reproduce the observed number of jets in data in the~\ttbar+jets dominated region. Therefore, CMS has created a custom tune where the~$\alpha_{\mathrm{ISR}}$ parameter, which controls the amount of initial-state radiation, and~$h_{\mathrm{damp}}$ parameter, which suppresses real emissions in~\powheg, have been adapted to reproduce the spectrum of the number of jets observed at CMS~\cite{CMS-PAS-TOP-16-021}. This MC tuning has been carried out on~$\sqrt{s} = 8~\TeV$~data and significantly improves the modelling of the jet multiplicity.

\begin{table}[h!]
\begin{center}
\begin{tabular}{cccc}
\hline
sample & generator & events & cross-section \\
\hline
\ttHbb\xspace & \powheg & 3~845~992 & 0.293~pb \\
\ttHnonbb & \powheg & 3~981~250 & 0.215~pb \\
\hline
\ttbar+jets (SL) & \powheg & 152~720~952 & 365.45~pb \\
\ttbar+jets (DL) & \powheg & 79~092~400 & 88.34~pb \\
\hline
\ttbar+W($\rightarrow \mathrm{\ell \nu}$) & \madgraphatnlo & 5~280~565 & 0.2043~pb \\
\ttbar+W($\rightarrow \mathrm{qq'}$) & \madgraphatnlo & 833~298 & 0.4062~pb \\
\ttbar+Z($\rightarrow \mathrm{\ell\ell, \nu\nu}$) & \madgraphatnlo & 13~908~701 & 0.2529~pb \\
\ttbar+Z($\rightarrow \mathrm{qq'}$) & \madgraphatnlo & 749~400 & 0.5297~pb \\
\hline
WW & \pythia & 7~981~136 & 118.7~pb~\cite{Gehrmann:2014fva} \\
WZ & \pythia & 1~988~098 & 47.13~pb \\
ZZ & \pythia & 3~995~828 & 16.523~pb \\
\hline
single top (tW) & \powheg, 4FS & 992~024 & 35.85~pb \\
single (anti)top (tW) & \powheg, 4FS & 998~276 & 35.85~pb \\
single top (t) & \powheg, 5FS & 5~993~676 & 136.02~pb \\
single (anti)top (t) & \powheg, 5FS & 3~928~063 & 80.95~pb \\
single top (s), $(\mathrm{W\rightarrow \ell\nu})$ & \madgraphatnlo, 4FS & 3~928~063 & 3.70~pb \\
\hline
W+jets $(\mathrm{W\rightarrow \ell\nu})$ & \madgraphatnlo & 29~705~748 & 61526.7~pb \\
$\mathrm{Z \rightarrow \ell \ell+jets}, m_{\ell\ell} < 50$~GeV & \madgraphatnlo & 35~291~566 & 18610~pb \\
$\mathrm{Z \rightarrow \ell \ell+jets}, m_{\ell\ell} \geq 50$~GeV & \madgraphatnlo & 145~803~217 & 5765.4~pb \\
\hline
\hline
\end{tabular}
\caption[The MC samples used in the~\ttHbb\xspace analysis]{The MC samples used in the~\ttHbb\xspace analysis. For the simulation of the parton shower and subsequent hadronization, \pythia\xspace is used for all samples, whereas \powheg\xspace or \madgraphatnlo\xspace are used for the hard process. The single top tW-channel process is simulated in the 4 flavour scheme (4FS), whereas other processes use the 5 flavour scheme (5FS) with a massless b~quark being present in the proton.}
\label{tab:mc_samples}
\end{center}
\end{table}

\begin{table}[h!]
\begin{center}
\begin{tabular}{c|cccc}
\hline
trigger & trigger threshold & data taking period & integrated luminosity \\
\hline
single muon & 22 GeV & Run B-H &~$35.9~\ifb$~\\
single electron & 27 GeV & Run B-H &~$35.9~\ifb$~\\
dimuon & 17 (8) GeV & Run B-H &~$35.9~\ifb$~\\
dielectron & 23 (12) GeV & Run B-H &~$35.9~\ifb$~\\
muon, electron & 23 (8) GeV & Run B-H &~$35.9~\ifb$~\\
\hline
\hline
\end{tabular}
\caption[The data samples used in the~\ttHbb\xspace  analysis]{The data samples.}
\label{tab:data_samples}
\end{center}
\end{table}

We list the details of the MC samples in~\cref{tab:mc_samples} and of data in~\cref{tab:data_samples}. In order to compare simulation to data, the simulated events are weighted according to the integrated luminosity and the predicted cross sections, which are taken from the best available inclusive calculations. In particular, the~\ttH~cross section is known at NLO accuracy~\cite{Dittmaier:1318996,Beenakker:2001rj,Beenakker:2002nc,Dawson:2002tg,Dawson:2003zu}. The Higgs boson branching fraction for~\Hbb~is affected by radiative corrections that are known up to N4LO (QCD) and NLO (electroweak), resulting in an uncertainty of about 1-2\%~\cite{Djouadi:1997yw,Butterworth:2010ym,deFlorian:2016spz}.
The cross section for~\ttbar+jets is know at NNLO accuracy and includes soft gluon resummation to next-to-next leading log (NNLL)~\cite{Czakon:2011xx}. The cross-sections of the minor backgrounds are known to at least NLO, as summarized in~\cref{tab:mc_samples}.

In addition to the hard interaction and the consequent showering and hadronization, events from additional pp interactions within the same bunch crossing (pileup) are superimposed on the simulated event for all processes. The multiplicity distribution of these additional pileup events is reweighted to match the observed number of interactions in data. Furthermore, we correct the MC simulation with additional data-driven correction factors for b-tagging and lepton efficiencies as described in~\cref{sec:systematic_unc}.

\fixme: PU plot

\subsection{Modelling of \ttbar+jets}
\label{sec:ttbar_subprocesses}
The main background in the \ttHbb search is the QCD production of~\ttbar+jets, for which we are using an NLO model based on \powheg. This process is affected by considerable scale uncertainties at the level of $\mathcal{O}(\alpha_s^4)$. Despite using an NLO model, at high jet multiplicities, the differential distributions are modelled only at LO or parton shower accuracy, depending on the momenta and multiplicity of jets. Therefore, the residual uncertainties in~\ttbar+heavy flavour production are significant in terms of the differential distributions, with different MC simulations having differences up to 100\% in variables sensitive to these effects. In general, to have an effective treatment of high-multiplicity events in a precise~\ttHbb\xspace measurement, an improved theoretical understanding of the underlying QCD process for $\mathrm{pp} \rightarrow \mathrm{t\bar{t}b\bar{b}}$, along with possible interference effects with the signal process is necessary~\cite{Denner:2014wka}.

In the absence of an experimentally verified model that accurately describes these high-multiplicity and heavy flavour processes, we account for these modelling uncertainties by assigning conservative uncertainties on the theoretical modelling. We subdivide the~\ttbar+jets sample based on the generator-level flavour of additional jets, as the underlying processes that give rise to these different flavour categories are different. In particular, it has been shown that \ttbb~can be significantly affected by the modelling of gluon splitting to collinear b quarks~\cite{Cascioli:2013era}. We distinguish between the following \ttbar+jets sub-processes:
\begin{itemize}
\item \ttbb, where two additional bottom jets are created from one or more b hadrons,
\item \ttb, with only one additional bottom jet, which can arise from a process with two b hadrons, with one b hadron being out of acceptance,
\item \tttwob, where jets from two b hadrons merge to produce one resolved bottom jet, which can arise from collinear gluon splitting,
\item \ttcc, if there are no additional bottom jets but at least one charm flavoured jet,
\item \ttlf (light flavour), in case there are no bottom or charm flavoured jets.
\end{itemize}
The jet flavour is assigned using so-called \textit{ghost clustering}, where simple geometrical matching between partons and jets is superseded by clustering the generator-level partons and hadrons along with the detector-level jet constituents using standard jet algorithms. Information from the generator-level decay chain is used assign the flavour of the jet according to the parton that gave rise to that jet~\cite{Bartosik:2047049}, such that only hadrons that did not arise from the decay chain of a top quark are used in this categorization. The aim of this splitting is to have experimental constraints on the uncertainties of the various~\ttbar~sub-processes separately, effectively relaxing some of the modelling assumptions. We illustrate the differences in the modelling of these processes on~\cref{fig:tth_ttjets_subprocesses} by comparing the distributions of leading jet transverse momentum ($p_T$) and the mass of the geometrically closest b jet pair $m_{\mathrm{b\bar{b}}}$, with significant differences between distributions visible for these processes.

\begin{figure}
\begin{centering}
\subfloat[Leading jet $p_T$]{\includegraphics[width=0.5\textwidth]{figures/tth/ttx_jet_pt.pdf}}
\subfloat[The invariant mass of the closest b jet pair]{\includegraphics[width=0.5\textwidth]{figures/tth/ttx_mbb.pdf}} \\
\caption[Modelling of the \ttbar+X sub-processes]{The modelling of the leading jet transverse momentum (\textbf{a}) and the invariant mass of the closest b jet pair (\textbf{b}) for the various \ttbar+jets sub-processes based on the \ttbar+jets POWHEG simulation from events with at least 2 jets with $p_T > 30$~GeV, $|\eta| < 2.5$.}
\label{fig:tth_ttjets_subprocesses}
\end{centering}
\end{figure}

\section{Event reconstruction and object identification}
\label{sec:object_id}
We use particle flow to reconstruct events from particle candidates based on signals from all sub-detectors, as described in~\cref{sec:particleflow}. This allows us to perform the analysis at the level of physics objects, namely, jets, charged leptons and \MET from neutrinos, which arise from the decay of the top quark pair. A top quark decays almost exclusively through $\mathrm{t} \rightarrow \mathrm{W}^+ \mathrm{b}$, such that the leptonic or hadronic decay of the W-boson, which happen with a branching ratio of 33\% or 67\% respectively, determines the final state. In the case of the semileptonic decay, which happens with a total branching ratio of 44\%, we expect 1 charged lepton ($\mathrm{e}^\pm, \mathrm{\mu}^\pm$), \MET from the corresponding neutrino, at least 6 jets, out of which 4 arise from bottom quarks and two from light quarks from the W-boson decay. In the dileptonic decay of the top quark pair, which happens with a branching ratio of $\simeq11\%$, we expect 2 charged leptons of opposite sign, \MET from neutrinos and at least 4 jets, all of which arise from bottom quarks. The fully hadronic decay of the top quark pair is treated in a separate analysis in CMS, as the background processes are significantly different between these cases. We note that the 2016 \ttHbb fully hadronic analysis relies on the same MEM implementation and general analysis code as this analysis, which were developed in a joint effort.

\subsection{Charged leptons}
\label{sec:object_id_lep}

The leptonic decay of at least one of the W-bosons from top decay is required in order to trigger the event at the HLT level. In order to suppress leptons from the multi-jet QCD background, the charged leptons are required to be sufficiently isolated from hadronic activity using an isolation variable, which is computed within a cone of radius~$\Delta R$~around the lepton direction (defined by the track) from the primary vertex as shown in~\cref{eq:iso_mu} for muons and~\cref{eq:iso_el} for electrons. In order to compute the isolation, we sum over the transverse momenta of all particle candidates ($p_T^{c.h.}$~for charged hadrons,~$E_T^{n.h.}$~for neutral hadrons,~$E_T^{\gamma}$~for photons) excluding the lepton itself and subtracting the neutral component from pileup events based on either the average pile-up energy ($\rho$) and effective area ($A$) for electrons or pile-up associated charged hadrons for muons. The pre-factor~$1/2$~for the pile-up component for muons is used to account for the approximate charged-to-neutral fraction in the hadronization of pile-up interactions~\cite{CMS:2012}.

Furthermore, in order to suppress leptons from non-prompt decays, we apply identification (ID) criteria based on various reconstruction parameters on the leptons. For muons, we apply the \textit{tight ID}, which is a cut-based selection that suppresses decays in flight and is based on properties of the global track fit, number of hits in the pixel detector, tracker and muon chambers and sufficient proximity to the primary vertex~\cite{Chatrchyan:2012xi,CMS:2017_muon_pog}. For electrons, the ID is based on a multivariate discriminator combining track-to-cluster matching observables, super cluster structure and cluster shapes~\cite{Khachatryan:2015hwa,CMS:2017_egamma_pog}.

We summarize the lepton selection criteria in all the considered channels in table~\cref{tab:lepton_selection} and describe the event selection in terms of leptons further in~\cref{sec:event_selection}.

\begin{equation}
\label{eq:iso_mu}
\mathrm{Iso}_{\mathrm{\mu}} = \sum_{\Delta R < 0.4} p_T^{c.h.} + \mathrm{max}\biggl(0, \sum_{\Delta R < 0.4} [E_T^{n.h.} + E_T^{\gamma} - \frac{1}{2} p_T^{\mathrm{PU}}] \biggr)
\end{equation}

\begin{equation}
\label{eq:iso_el}
\mathrm{Iso}_{\mathrm{e}} = \sum_{\Delta R < 0.3} p_T^{c.h.} + \mathrm{max}\biggl(0, \sum_{\Delta R < 0.3} [E_T^{n.h.} + E_T^{\gamma} - \rho A(\eta)] \biggr)
\end{equation}

\begin{table}[h!]
\begin{center}
\caption{The selection and ID criteria for the charged leptons.}
\label{tab:lepton_selection}
\begin{tabular}{c|ccccc}
\hline
channel & trigger & offline~$p_T$~&~$|\eta|$~& isolation \\
\hline
$\mathrm{\mu}^\pm$~&~$p_T > 22\GeV$~&~$p_T > 25\GeV$~&~$|\eta| < 2.1$~& ~$\mathrm{Iso}/p_T < 0.15$~\\

$\mathrm{e}^\pm$~&~$p_T > 27\GeV$~&~$p_T > 30\GeV$~&~$|\eta| < 2.1$~&~$\mathrm{Iso}/p_T < 0.15$\\

$\mathrm{e}^\pm\mathrm{e}^\mp$~&~$p_T > 23 (12)\GeV$~&~$p_T > 25 (15)\GeV$~&~$|\eta| < 2.1$~&~$\mathrm{Iso}/p_T < 0.15$\\

$\mathrm{e}^\pm\mathrm{\mu}^\mp$~&~$p_{T} > 23_{\mathrm{e}} (8_{\mathrm{\mu}})\GeV$~&~$p_T > 25 (15)\GeV$~&~$|\eta| < 2.4$~&~$\mathrm{Iso}/p_T < 0.25_{\mathrm{\mu}} (0.15_{\mathrm{e}})$~\\

$\mathrm{\mu}^\pm\mathrm{e}^\mp$~&~$p_{T,} > 23_{\mathrm{\mu}} (8_{\mathrm{e}})\GeV$~&~$p_T > 25 (15)\GeV$~&~$|\eta| < 2.4$~&~$\mathrm{Iso}/p_T < 0.25_{\mathrm{\mu}} (0.15_{\mathrm{e}})$~\\

$\mathrm{\mu}^\pm\mathrm{\mu}^\mp$~&~$p_T > 17 (8)\GeV$~&~$p_T > 25 (15)\GeV$~&~$|\eta| < 2.4$~&~$\mathrm{Iso}/p_T < 0.25$~\\

\hline
\hline
\end{tabular}
\end{center}
\end{table}

\subsection{Jets}
\label{sec:object_id_jets}
As the signal process is expected to produce between 4 to 6 jets in the leading order description and additional jets due to QCD radiation and pileup, an accurate reconstruction of jets is critical for this analysis. We use the anti-$k_T$~clustering algorithm~\cite{Cacciari:2008gp} in the \texttt{FASTJET} implementation~\cite{Cacciari:2011ma} with a distance parameter $\Delta R=0.4$ to reconstruct jets from particle flow candidates~\cite{CMS:2010xta,CMS:2009nxa,CMS:2010byl} and use the CMS PF jet ID algorithm to reject reconstruction failures and noise. The noise rejection works on the basis of cuts on jet energy fractions from various types of PF candidates, namely muons, electrons, photons, charged hadron and neutral hadron candidates and has a noise rejection of around 99\%~\cite{CMS:2017wyc}.

\subsubsection{Charged hadron subtraction}
Charged particles from pileup interactions are removed from clustering via the process of charged hadron subtraction (CHS). As CHS relies on the reconstruction of tracks and the association of charged hadrons to tracks, the procedure is applied within the tracker volume ($|\eta| < 2.5$). We choose the leading primary vertex (PV) as the one that has the highest magnitude of total track transverse momentum squared ($\sum |p_T^{\mathrm{track}}|^2$), with the rest of the PVs passing certain quality criteria as subleading PVs. Charged hadrons that are associated to tracks that are compatible with subleading PVs are removed. The subtraction procedure reduces the amount of jets arising from pileup from about 20\% to 5\% in the tracker region and also improves the momentum resolution and angular resolution ($\Delta R \simeq 0.01$) of jets~\cite{CMS:2014ata}.

\subsubsection{Jet energy scale calibration}
\label{sec:jes_calibration}
The experimentally measured energies of the jets have to be calibrated in terms of jet energy scale (JES) and resolution (JER) in both data and simulation. This is done using jet energy corrections (JEC), which correct for offset energy from pileup, the detector response based on simulation, the residual differences between data and simulation based on well-understood channels such as di-jet production, and the detector response to jet flavour. 

The presence of pileup interactions generates a diffuse energy component that results in an energy offset in the jets. This offset correction is estimated using simulation by comparing jets in a MC sample without pileup events to the same simulation overlayed with pileup, geometrically matching them to the same underlying jet on the generator level~\cite{cms_jec_2017}. An additional scale factor between data and simulation is extracted from zero-bias data using the random cone method~\cite{Chatrchyan:2011ds}.  

The detector response is defined as the ratio between the reconstructed jet and a geometrically matched particle-level jet, averaged over a sample of jets:~$R = \langle p_T \rangle / \langle p_{T,\mathrm{particle}} \rangle$. It is estimated using a detailed model of the detector geometry, alignment, calibration and electrons, implemented in \texttt{GEANT4} in bins of particle-level jet momentum and reconstructed jet~$\eta$. The corrections are able to bring the response to a deviation of approximately 1\% from unity based on simulations. A residual data to simulation correction scale factor is applied on data based on transverse momentum balance from dijet,~$\mathrm{Z}/\gamma$+jets and multi-jet events, with the momentum projection fraction (MPF) with respect to the missing transverse momentum being used as an alternative derivation for the corrections. These relative corrections rely on comparing the jet under calibration (probe) to a reference object (tag) and are of the order of a few percent in the central region considered in this analysis~\cite{Chatrchyan:2011ds,cms_jec_2017}.

The jet response to different flavours is estimated using simulation, comparing the response of jets associated with partons according to a geometric matching between~\pythia~and~\herwig. The magnitude of the flavour response is generally within a few percent, with differences between the flavours arising from fragmentation, where gluons fragment the most into soft particles that may remain unreconstructed and thus have the lowest response, and particle composition, with the neutral hadronic component having the largest effect. The flavour corrections are validated in Z+b~jet data and the residual correction between data and simulation is found to be consistent with unity~\cite{Chatrchyan:2011ds}.

\subsubsection{Jet energy resolution}
In contrast to jet energy scale, which is known with a total uncertainty better than~$3\%$~over the relevant phase space, the jet energy resolution is known to around~$10-20\%$. The resolution can be determined using~$p_T$~balance as for JES, but measuring the width instead of the mean of the response distribution. Both~$\mathrm{Z}/\gamma$+jet and dijet events are used to determine the JER response~\cite{Chatrchyan:2011ds}. In the construction of the MEM, we also use approximate jet resolution functions derived from simulation in order to account for detector effects in the phase space integral as explained in~\cref{sec:transfer_functions}.

\subsection{B-tagging}
\label{sec:object_id_btag}

Since the \ttHbb\xspace signal is characterized by the presence of 4 bottom quarks in the final state, with two arising from the top quarks and two from the Higgs boson decay, an accurate identification of b~jets arising is important in this analysis. We rely on the combined secondary vertex algorithm (CSVv2)~\cite{Chatrchyan:2012jua} to identify b jets. The CSVv2 algorithm uses secondary vertex properties such as the impact parameter along with track-based lifetime information to create a robust combined discriminator~$\xi$~optimised to distinguish between jets arising from bottom quarks and light quarks using supervised learning. In Run 2 of the LHC, the CSVv2 algorithm has been improved with a new vertex reconstruction algorithm, as well as using artificial neural networks instead of a likelihood method to combine the input variables, such that correlations between the inputs are taken into account~\cite{CMS-PAS-BTV-15-001}.

The threshold value of the b-tagging discriminant, above which a jet is considered to be b-tagged is chosen such that the efficiency of misidentifying jets arising from light quarks (u,d,s) or gluons as b~jets would be sufficiently low (~$1\%$). This corresponds to the efficiency of~$70\%$~of correctly identifying bottom quarks and of mis-identifying charm quarks around~$20\%$~and is denoted the CSVv2 medium working point (CSVM).

\subsubsection{B tagging likelihood}
We further use the value of the per-jet b-tagging discriminant~$\xi$~to construct a per-event likelihood discriminator between the hypotheses that the event contained 4 bottom quarks (``$4\mathrm{b}$'') or 2 bottom quarks (``$2\mathrm{b}$'') as shown in~\cref{eq:blr}. The sum is performed over all the combinations of associating~$M$~jets out of~$N$~to bottom quarks and the rest to light quarks and~$\mathrm{b}_i$~($\mathrm{l}_i$) refers to the~$M$~($N-M$) jets associated to bottom quarks (light quarks) in the~$i$-th combination. We have used the function~$f(\xi_k | \mathrm{b})$~($f(\xi_k | \mathrm{l})$), which is the probability density that the~$k$-th jet has a discriminator value~$\xi_k$~assuming that it originated from a bottom quark (light quark). These b-tagging likelihoods~$\mathcal{BL}(\vec{\xi} | M\mathrm{b})$~are then used to construct a likelihood ratio discriminator~$\mathcal{BLR}(\vec{\xi})$~(\cref{eq:blr_ratio}) that is optimised to suppress the the~\ttlf~background with two bottom quarks in favour of the~\ttHbb\xspace signal with 4 bottom quarks.

\begin{equation}
\label{eq:blr}
\mathcal{BL}(\vec{\xi} | M\mathrm{b}) = \sum_{i \in \mathrm{perm}} \biggl[ \prod_{k \in \mathrm{b}_i} f(\xi_k | \mathrm{b}) \prod_{k \in \mathrm{l}_i} f(\xi_k | \mathrm{l}) \biggr]
\end{equation}

\begin{equation}
\label{eq:blr_ratio}
\mathcal{BLR}(\vec{\xi}) = \frac{\mathcal{BL}(\vec{\xi} | 4\mathrm{b})}{\mathcal{BL}(\vec{\xi} | 4\mathrm{b}) + \mathcal{BL}(\vec{\xi} | 2\mathrm{b})}
\end{equation}

We assess the performance of this discriminator in terms of \ttHbb\ vs.~\ttlf~discrimination in simulation. On~\cref{fig:blr_discrimination}, we see that the~$\mathcal{BLR}$~discriminant improves over a fixed cut of~$\ge4$~jets passing the CSVv2 medium working point ($N_{\mathrm{CSVM}} \ge 4$) by about 50\% ($\epsilon_{\ttlf} = 0.04\% \rightarrow 0.022\%$) in terms of background rejection at the same signal efficiency ($\epsilon_{\ttHbb} \simeq 7\%$).

Furthermore, we have studied the efficiency of~$\mathcal{BLR}$~in correctly reducing the number of permutations in the MEM by assigning jets to be b-tagged or untagged. For this, we evaluated the fraction of events where the final jets can be correctly matched to quarks from the hard interaction as a function of~$\mathcal{BLR}$. We see on~\cref{fig:blr_matching} that the likelihood discriminator is positively correlated with the probability that all the quarks have been matched to jets, where around~$50\%$~of bottom quarks from top decay, $40\%$~of bottom quarks from Higgs decay and around~$20\%$~of the light quarks from the W boson decay have been reconstructed as jets at~$\mathcal{BLR} \simeq 0.8$. Furthermore, we see a positive correlation between the likelihood discriminator and the probability that the highest-probability permutation in the sum in~\cref{eq:blr} with the~$4$~bottom quark hypothesis corresponds to the bottom quarks from top or Higgs decay. In other words, the likelihood discriminator successfully tags the bottom quarks on an event-by-event basis.

\begin{figure}
\begin{centering}
\subfloat[Simulated shape of the discriminant.]{\includegraphics[width=0.5\textwidth]{figures/blr_shape_btagCSV.pdf}} 
\subfloat[Expected performance of the discriminant.]{\includegraphics[width=0.5\textwidth]{figures/blr_roc.pdf}}\\
\caption[Expected performance of the b-tag likelihood ratio discriminant.]{Simulated distribution and expected performance of~$\mathcal{BLR}$~discriminant in the SL channel, requiring at least 4 good jets. On~(\textbf{a}), we show the simulated shapes of the discriminant for signal~(\ttHbb) and the various \ttbar+jets backgrounds. On~(\textbf{b}), we compare the efficiency to select~\ttHbb\xspace and~\ttlf~events. We see that the~$\mathcal{BLR}$~discriminant compares favourably to a fixed cut on number of b-tags. The~$\mathcal{BLR}$~ discriminator defined with the cMVAv2 b-tagger algorithm further improves the performance over the full range.}
\label{fig:blr_discrimination}
\end{centering}
\end{figure}


\begin{figure}
\begin{centering}
\subfloat[Fraction of events with correct matching.]{\includegraphics[width=0.5\textwidth]{figures/blr_matching.pdf}} 
\subfloat[Fraction of matched events with correct tagging.]{\includegraphics[width=0.5\textwidth]{figures/blr_matching_tag.pdf}}\\
\caption[Fraction of events with correct matching to the hard process]{Estimation of the fraction of events where the bottom quarks from the top quark, the Higgs boson and the light quarks from the W boson are reconstructed as jets in the final state on (\textbf{a}). We study the event-level b-tagging efficiency on (\textbf{b}), where we plot the fraction of events where the highest-likelihood permutation correctly assigned the bottom quarks to jets with respect to all events where the quarks were reconstructed as jets without considering tagging.}
\label{fig:blr_matching}
\end{centering}
\end{figure}
 
The likelihood ratio as defined here ignores the differences in~$f(\xi_k | \mathrm{b})$~and~$f(\xi_k | \mathrm{c})$, meaning that in the case of~$\mathrm{W} \rightarrow \mathrm{c}\bar{\mathrm{s}} (\bar{\mathrm{d}})$~decays, the discriminator is suboptimal. We have investigated extending this likelihood to also account for the possibility of such decays by a straightforward extension of~$\mathcal{BL}(\vec{\xi} | M\mathrm{b}~1\mathrm{c})$. However, we have found that the additional combinatorial complexity suppresses any increased discrimination power and further progress would likely require methods that are better able to deal with the combinatorial problem using kinematic information. We use this b-tagging likelihood ratio as a discriminator between the various~\ttbar+jets sub-processes, as well as to select the bottom quark candidates in the application of the MEM, as described in~\cref{sec:mem_application}. We have additionally studied whether switching to the new combined multivariate b-tagging algorithm (cMVAv2) introduced in~\cref{sec:btagging} would improve the analysis. Since the irreducible \ttbb\xspace background contains the same number of b~jets as the \ttHbb\xspace signal, improved b-tagging can only reduce contribution from the background components with light quarks, which are less problematic. We find that although there is a small increase in the signal rate by switching the b~discriminator, it is not sufficient to motivate a re-optimization of the analysis at this stage. 

\subsubsection{B~discriminator shape calibration}
As we have used the detailed b~discriminator shape information in constructing the b-tagging likelihood ratio, we must experimentally calibrate the full range of this observable using data. This is accomplished by deriving a reweighting factor between data and simulation that depends on the jet b~discriminator value, jet kinematics and flavour using a tag-and-probe method. In this approach, the tag jet is required to pass the medium operating point that has been described earlier and the discriminator distribution of the probe jet is corrected by reweighting the MC simulation. In order to extract the weight for the b jets, the procedure relies on dilepton~\ttbar+jets events with the contribution from light jets and backgrounds subtracted using simulation, whereas for the scale factor for light jets, Z+jets events are used. The procedure is iterative, as the scale factor for light jets is required for the extraction of the b jet scale factor and vice versa~\cite{CMS:2013sea,CMS-PAS-BTV-15-001}. The systematic uncertainties from this reweighting method are described in~\cref{sec:systematic_unc}.

\subsubsection{Missing transverse energy}
The leptonic decays of the W boson produce neutrinos, which are only partially reconstructed by the detector as \MET, defined as the negative sum of all the momenta of the reconstructed particles in the transverse plane. In the SL channel, we can directly associate the~\MET~with the transverse momenta of the neutrino through the modelling of the recoil as described in~\cref{sec:transfer_functions} whereas in the DL channel, only the total momentum of both neutrinos is constrained by the \MET.

\section{Analysis}
\label{sec:analysis}
\subsection{Event selection and categorization}
\label{sec:event_selection}

First, the large multi-jet QCD background is reduced to negligible levels by requiring that at least one of the top quarks in~\ttHbb\xspace process decays leptonically. We divide events to two exclusive lepton categories: SL and DL, based on the multiplicity of the reconstructed charged leptons passing the quality cuts described in~\cref{sec:object_id_lep}. This is achieved by vetoing events with any additional leptons passing loosened quality criteria. We further suppress the Drell-Yan background by requiring~$m_{\ell\ell} > 20\GeV$~and Z+jets in the DL categories by rejecting events around the resonant Z peak with~$76\GeV < m_{\ell\ell} < 106\GeV$. Furthermore, the leptons are required to have opposite charge. In the DL same-flavour categories, we require~$\MET > 40\GeV$. We do not explicitly distinguish between cases where the top quark decays to~$\mathrm{\tau}$~leptons, although these events can still pass the selection in case the~$\mathrm{\tau}$~decays leptonically and are considered as signal.

Events from \ttHbb\xspace have a large number of jets and b-tags compared to the V+jets backgrounds. Therefore, we require the presence of at least 4 jets passing the quality criteria (\cref{sec:object_id_jets}), out of which at least 3 must be b-tagged according to the medium working point (\cref{sec:object_id_btag}). This brings us to the~\ttbar~dominated region, where we further distinguish between 6 categories in the SL channel
\begin{itemize}
\item~$\geq$6 jets, $\geq$4 b-tags;\ 5 jets,\ $\geq$4 b-tags and 4 jets, 4 b-tags, that are the most signal-enriched categories,
\item~$\geq$6 jets, 3 b-tags;\ 5 jets, 3 b-tags and 4 jets, 3 b-tags, that contain a significant amount of \ttcc~and \ttbb,
\end{itemize}
and two categories in the DL channel
\begin{itemize}
\item~$\geq$ 4 jets, $\geq$ 4 b-tags,
\item~$\geq$ 4 jets, 3 b-tags,
\end{itemize}
resulting in a total of 8 exclusive categories, shown on~\cref{fig:tth_pies}.

\begin{figure}
\begin{centering}
\includegraphics[width = 1.0\textwidth]{figures/tth/pies.pdf}
\caption[The expected signal and background yields in the analysis categories]{The expected signal and background yields in the semileptonic (SL) and dileptonic (DL) analysis categories. We also show the expected signal over background ratio $S / \sqrt{B}$. As can be seen, the expected signal yield is quite low compared to the predicted background, even in the most high-purity categories of SL $\geq$6jet, $\geq$4tag and DL $\geq$4jet, $\geq$4tag.}
\label{fig:tth_pies}
\end{centering}
\end{figure}

We use the MEM as a \ttHbb\xspace to~\ttbb~discriminator in the categories with~$\ge 4$~b-tags as these categories are enhanced in the signal fraction and have been shown to have an excellent discriminator performance for the MEM. The categories with 3 b-tags are retained as control regions where we determine the \ttbar+jets background rates by fitting the b-tagging likelihood discriminator. We now turn to the description of the signal extraction using a template fit. 

\subsection{Signal extraction}
\label{sec:mem_application}
The likelihood discriminant based on b-tagging enhances the~\ttbar+heavy flavour component, but the cross-section of~\ttbb~is still an order of magnitude larger that of \ttHbb. Furthermore, we cannot directly reconstruct the resonant peak of the \Hbb~decay as a natural discriminant between the signal and non-resonant background. Even though the width of the SM Higgs boson is relatively small compared to detector resolution ($\Gamma_{\mathrm{SM}} = 4.07 \times 10^{-3}~\GeV$), the presence of multiple additional bottom quarks due to top decay in the final state creates a combinatorial self-background in the form of an ambiguity in choosing the candidate jets for the \Hbb~decay.

An estimator for the Higgs candidate invariant mass, built from randomly selected jet pairs, results in a much broader distribution compared to experimental resolution, whereas choosing the pair of jets that would give a mass closest to~$m_H$~would cause also the background to exhibit a signal-like peak.

Therefore, we use the MEM discriminant, introduced in~\cref{sec:mem}, to compute theory-motivated weights~$P_{\ttHbb}$~and~$P_{\ttbb}$~for each candidate event. We construct a signal-to-background discriminant~$P_{\mathrm{s/b}}$~based on the likelihood ratio of these weights, as described in~\cref{sec:mem_performace}, which based on the Neyman-Pearson lemma, described in~\cref{sec:test_statistic}, is the optimal test statistic between the signal and background hypotheses.

As an improvement over the search for~\ttHbb\xspace performed by the CMS experiment in Run I~\cite{Khachatryan:2015ila}, we use the MEM discriminant also in categories which are not fully reconstructed, but still contain a large amount of signal, namely 5-jet and 4-jet categories in the SL channel. The details of the additional MEM hypotheses are described in~\cref{sec:event_interpretation}. We list the discriminants that have been used in the different categories in~\cref{tab:cat_discriminant}. 


\begin{table}[h!]
\begin{center}
\caption[The analysis categories for the~\ttHbb\xspace analysis.]{The analysis categories and the discriminators used in those categories.}
\label{tab:cat_discriminant}
\begin{tabular}{c|c}
\hline
category & discriminant \\
\hline
SL~$\geq6$~jets,~$\geq4$~tags & MEM SL~$2_{\mathrm{W}} 2_{\mathrm{h}} 2_{\mathrm{t}}$~\\
SL~$\geq6$~jets,~$3$~tags & The b-tagging likelihood ratio \\
\hline
SL~$5$~jets,~$\geq4$~tags & MEM SL~$1_{\mathrm{W}} 2_{\mathrm{h}} 2_{\mathrm{t}}$~\\
SL~$5$~jets,~$3$~tags & The b-tagging likelihood ratio \\
\hline
SL~$4$~jets,~$\geq4$~tags & MEM SL~$0_{\mathrm{W}} 2_{\mathrm{h}} 2_{\mathrm{t}}$~\\
SL~$4$~jets,~$3$~tags & The b-tagging likelihood ratio \\
\hline
DL~$4$~jets,~$\ge4$~tags & MEM DL~$2_{\mathrm{h}} 2_{\mathrm{t}}$~\\
DL~$4$~jets,~$3$~tags & MEM DL~$2_{\mathrm{h}} 2_{\mathrm{t}}$~\\
\hline
\hline
\end{tabular}
\end{center}
\end{table}

We use the~\ttbb~matrix element as a representative background diagram in all categories. This gives the best separation in the most signal-rich categories against~\ttbb~and is further motivated by simulation, where we see that using this process as background still achieves a high rate of separation in categories enriched in other \ttbar+jets sub-processes. As an optimization, considering additional background hypotheses in different categories in the future is expected to improve the signal-to-background discrimination at the cost of computational complexity.

As the b-tagging likelihood ratio method is optimised to identify the set of jets that are most compatible with arising from 4 bottom quarks, we further augment the MEM by assuming that the bottom quarks need to be considered only among those 4 jets, as explained in~\cref{sec:event_interpretation}. This means that the we have exactly 4 candidates for the bottom quarks from \Hbb~and~$\mathrm{t} \rightarrow \mathrm{W}^+ \mathrm{b}$~decay, whereas the remaining jets are assumed to arise from~$\mathrm{W} \rightarrow \mathrm{q} \mathrm{q}'$~or from unspecified sources. 

Both of these discriminators are complex multivariate functions based on quantities which have significant modelling uncertainties affecting the shapes of the distributions. Therefore, a realistic description and propagation of the systematic uncertainties is crucial in the interpretation of data.

\subsection{Systematic uncertainties}
\label{sec:systematic_unc}
Among the experimental uncertainties, the dominant ones are uncertainties on the jet energy scale and resolution corrections (\cref{sec:jec_unc}) and the reweighting of the b-tagging discriminant (\cref{sec:btag_unc}). Both of these can affect the predicted yields of all the processes, since they change the selection efficiency, as well as the shapes of the final discriminants. In case the source of an uncertainty is the same across several categories, the nuisance parameters associated with the uncertainties are treated as fully correlated, otherwise, the uncertainties are treated as uncorrelated in the fit.

From the theoretical uncertainties, the uncertainties on the modelling of the \ttbar+jets background are the ones with the largest impact. As introduced in~\cref{sec:ttbar_subprocesses}, the~\ttbar+jets \texttt{POWHEG} model we currently use in the analysis treats the~\ttbb~process only at leading order accuracy, where the~$\mathrm{b}\bar{\mathrm{b}}$~pair is generated from gluon splittings using a parton shower, such that we have considerable additional theoretical uncertainties on the modelling of \ttbb, as will be described in~\cref{sec:theory_unc}.

We give a detailed overview of the most important experimental and theoretical uncertainties along with their estimation in the next sections. The full list of systematic uncertainties along with the assumed priors is show in~\cref{tab:systematic_uncertainties_prior}.

\begin{table}[h!]
\begin{center}
\begin{tabular}{c|cccc}
\hline
uncertainty & normalization & shape & processes & prior \\
\hline
JES (26 sources) & yes & yes & all & Gaussian, $<5\%$ \\
JER & yes & yes & all & Gaussian, $<1\%$ \\
b-tagging (9 sources) & yes & yes & all & Gaussian, $0-20\%$ \\
pileup & yes & yes & all & Gaussian, $0-5\%$ \\
lepton ID & yes & yes & all & Gaussian, $\simeq1\%$ \\
lepton isolation & yes & yes & all & Gaussian, $\simeq1\%$ \\
luminosity & yes & no & all & log-normal, $2.4\%$ \\
limited MC statistics & no & yes & all & bin-by-bin Poisson \\
\hline
\ttbar+jets ISR, FSR & yes & partly & \ttbar+jets & log-normal, 0-15\% \\
\ttbar+jets tune, $h_{\mathrm{damp}}$ & yes & partly & \ttbar+jets & log-normal, 0-15\% \\
\ttbb~norm. & yes & no & \ttbar+jets & log-normal, 50\% \\
\ttb~norm. & yes & no & \ttbar+jets & log-normal, 50\% \\
\tttwob~norm. & yes & no & \ttbar+jets & log-normal, 50\% \\
\ttcc~norm. & yes & no & \ttbar+jets & log-normal, 50\% \\
PDF~(gg) & yes & no & \ttH, \ttbar+jets & log-normal, 4\% \\
PDF~(qq') & yes & no & W+jets & log-normal, 2\% \\
PDF~(qg) & yes & no & single top & log-normal, 3\% \\
$\mu_R$, $\mu_F$ scale & yes & yes & \ttbar+jets & Gaussian \\
scale uncertainties in norm. & yes & no & primarily \ttH & log-normal \\
\hline
\hline
\end{tabular}
\caption[Systematic uncertainties in the~\ttHbb\xspace analysis.]{Systematic uncertainties in the~\ttHbb\xspace analysis.}
\label{tab:systematic_uncertainties_prior}
\end{center}
\end{table}

\subsection{Jet energy correction uncertainties}
\label{sec:jec_unc}
We apply jet energy scale (JES) and resolution (JER) corrections between data and simulation, as described in~\cref{sec:jes_calibration}. Thus, we need to understand the effect of the uncertainties on these corrections. In Run II, we consider various uncorrelated sources of jet energy scale correction uncertainties with their corresponding correlations, as opposed to a single bulk JES uncertainty as was done in this analysis in Run I. This significant advancement has resulted from an improved modelling of the detector performance and better calibration techniques developed with more data. By treating the various sources of JES uncertainties independently, the assumptions that allow the profile likelihood fit to constrain the combined JES uncertainty significantly in Run I are thus relaxed and the final uncertainty is a more realistic estimate of the true uncertainty.

The magnitude and correlation of the uncertainties on JES and JER are determined in a dedicated CMS analysis and are provided as a vector of per-jet corrections with~$p_T$~and~$\eta$~dependent correlations~\cite{cms_jec_2017}. The most important groups of correction uncertainties are the following:

\begin{itemize}
\item Pileup offset, which results from from extra energy deposited in jets from additional pp interactions within the same bunch crossing (in-time pileup) or due to the finite signal decay time in the calorimeters (out-of-time pileup). The uncertainty for this source results from the~$\eta$-dependent scale factor used to correct the offset distribution measured in simulation.
\item Relative~$\eta$-dependent corrections, which calibrate the forward regions of the detector with respect to the central region. Uncertainties of this type arise from jet energy resolution and from the modelling of initial and final state radiation (ISR+FSR).
\item Uncertainties on the absolute energy scale, which are derived using Z/$\gamma$+jet and multijet data. The energy scale uncertainties are driven by the muon momentum scale and the single pion response in the HCAL. Furthermore, the uncertainties in fragmentation are assessed in a comparison of \texttt{PYTHIA} and \texttt{HERWIG++} MC models.
\item Uncertainties in the modelling of the detector response for jet flavour, which are assessed using simulation and are largest for gluon jets.
\item Finally, due to radiation damage, there is a residual time-dependent uncertainty in the scale corrections, which is estimated using dijet events in different run periods.
\end{itemize}

These 5 broad groups factorize into approximately 26 independent sources. In order to account for the JES scale uncertainties in the analysis, we propagate the uncertainties in jet energy scale and resolution corrections to the jet momenta and all the event-level observables that are derived from them, such as the jet and b-tag multiplicities, by changing the jet energy scales and resolutions by one standard deviation up and down form the nominal values. This is done separately for all the sources so that we can fully account for the correlations between the various sources. Thus, we are able to account for both the changes in efficiency (normalization) and discriminator shape in the final analysis categories. We find that the normalization effects are of the order of 0-4\% for all JES and JER sources for the SL channel and around 0-1\% in the DL channel, with the largest variations resulting from the jet flavour response modelling.

In order to propagate the uncertainty to high-level multivariate observables such as the MEM, they need to be recomputed using the variated jets. We use the approximate MEM vector integration technique for this as described in~\cref{sec:mem_uncertainties}. In addition to uncertainties in the jet energy scale corrections, we also consider uncertainties on the jet energy resolution corrections. In the tracker region, the JER uncertainty is around 2-4\%, depending on the jet pseudorapidity $|\eta|$. The JER uncertainty is propagated by shifting the JER scale factor up and down by one standard deviation corresponding to the uncertainty, thus it is fully deterministic. The overall effect of the JES and JER uncertainties on the leading jet transverse momentum distribution is shown on~\cref{fig:jec_pt_effect}. These uncertainties will mostly affect the predicted yields in our final analysis categories.

\begin{figure}
\begin{centering}
\includegraphics[width=1.0\textwidth]{figures/tth/sl_jge4_tge2_jec_unc.pdf}
\caption[Effect of jet energy corrections.]{The effect of jet energy corrections on the leading jet transverse momentum distribution. In the top row, we show the distribution under variations of the flavour composition (left), jet energy resolution (middle) and the statistical uncertainties in the absolute scale determination (right). In the middle row, we show the overall uncertainty from the absolute scale variation (left), the uncertainty arising from the corrections derived using the missing momentum projection fraction (MPF, middle) and the time-dependent momentum scale variation (right). On the bottom row, we show the uncertainties in the single pion response in the ECAL (left), HCAL (middle) and the fragmentation model (right). In general, we see that the variations are within a few percent of the nominal, with the largest effect from the uncertainties on the flavour response (top left).}
\label{fig:jec_pt_effect}
\end{centering}
\end{figure}

\subsection{B-tagging systematic uncertainties}
\label{sec:btag_unc}
Due to the high expected number ($\geq4$ in the signal regions) of b-tagged jets in the~\ttHbb\xspace final state, this analysis is sensitive to uncertainties in b-tagging. As we have described in~\cref{sec:object_id_btag}, we correct for mis-modelling in the b-tagging discriminator shape using a tag and probe method, such that the detailed b~discriminator shape can be used in further template fitting. The effect of this shape correction is shown on~\cref{fig:tth_btag_rew}, where we see that the correction improves the description by the MC, with the corrected distribution agreeing with data within the systematic uncertainties.

The uncertainties of this correction include the propagation of jet energy scale uncertainty, which affects the determination of the correction through changes in efficiency and the discriminator value. Furthermore, simulation is used to subtract the non-relevant jet favour components in determining the scale factor for bottom (light) jets. For the scale factor for light jets, the fraction of bottom (charm) jets is variated within 20\% of the MC prediction in the Z+jets simulation used to determine the scale factors. Similarly, for the extraction of the b jet scale factor, the light flavour component in the \ttbar+jets dileptonic sample arises from additional radiation and is estimated to be 20\%~\cite{CMS-PAS-BTV-15-001}.

As the b discriminator scale factor is determined in bins of the discriminator value, statistical fluctuations in bins with a low number of data and simulated events introduce an uncertainty on the final scale factor. This uncertainty is only significant in case the size of the fluctuations varies systematically over the b discriminant range. Since the discriminator has a roughly monotonous increasing (decreasing) shape for b jets (light jets), this condition is fulfilled. The statistical uncertainties are accounted for by a sum of polynomials of first and second order, where the nuisance parameter is the overall scale of the distortion.

There is currently no dedicated scale factor for the b discriminator of charm jets, therefore, the uncertainties on the charm flavour scale factor are assumed to be twice as large as for the b jet scale factor. We propagate the uncertainties from b-tagging in the form of a set of per-event weights, which are determined from the individual per-jet weights that are used to correct the jet b discriminator distributions. The uncertainties on the b~discriminator scale factor result in both normalization effects due to acceptance changes, which can be up to 20\% in some cases, and shape effects on the templates. An example of the effect of b~discriminator uncertainties is shown on~\cref{fig:tth_btag_unc}. 

\begin{figure}
\begin{centering}
\subfloat{\includegraphics[width = 0.5\textwidth]{figures/tth/sl_jge4_tge2_csv.pdf}}
\subfloat{\includegraphics[width = 0.5\textwidth]{figures/tth/sl_jge4_tge2_csv_rel.pdf}}
\caption[B-tagging reweighting and uncertainty]{The effect of the b-tag reweighting on the MC modelling (left). We see that the b-tag reweighting technique improves the modelling from the nominal case (red) to the corrected case (blue), with the total uncertainties (pink) covering the difference. On the right plot, we see that the overall uncertainty band consists mainly of the light flavour (blue), charm flavour (green) and heavy flavour (orange) uncertainty components.}
\label{fig:tth_btag_rew}
\end{centering}
\end{figure}


\begin{figure}
\begin{centering}
\includegraphics[width = 1.0\textwidth]{figures/tth/sl_jge6_t3_blr_unc.pdf}
\caption[The effect of b-tagging uncertainties on the b-tagging likelihood ratio.]{The effect of b-tagging uncertainties on the b-tagging likelihood ratio distribution for the \ttHbb~sample. In the top row, we show the effect of the heavy flavour (left), light flavor (center) and the linear heavy flavour distortion from statistical uncertainties (right). In the middle row, we show the effect of the quadratic distortion heavy flavour scale factor distortion from statistical uncertainties (left) and the linear and quadratic uncertainties on the light flavour scale factor (middle, right). In the bottom row, we show the uncertainties for the charm flavour jets (left, middle) and the uncertainty on the scale factor arising from the propagation of JES uncertainties. These templates are derived in the SL $\geq6$ jet, 3 b-tag category on \ttH simulation.}
\label{fig:tth_btag_unc}
\end{centering}
\end{figure}

\subsection{Other experimental systematic uncertainties}
We also assess the effect of uncertainties in the lepton identification, isolation and trigger selection, which may have different efficiencies in data and simulation and are thus corrected using scale factors. For muons, we assign a~$1\%$~normalization uncertainty for the lepton ID,~$1\%$~for isolation and~$0.5\%$~for the so-called HIP effect, on top of the statistical uncertainties on the muon scale factor~\cite{CMS:2017_mu_sf}. For electrons, we use~$p_T$~and supercluster~$\eta$-dependent scale factor uncertainties derived using a tag-and-probe method, which are generally below~$1\%$~\cite{CMS:2017_ele_sf}.

As the pileup profile in simulation is corrected to data using a pileup-dependent scale factor, we estimate the uncertainty in the pileup correction by varying the minimum bias cross section from~$\sigma = 69.2$~mb by~$4.6\%$, corresponding to the uncertainty in the number of interactions in minimum bias events from luminosity and cross section\cite{CMS:2017_pu_weight_twiki}. This results in both a normalization and shape effect in the final templates.

Furthermore, the uncertainty on total integrated luminosity is estimated to be~$2.5\%$~using cluster counting in the pixel detector and affects all processes~\cite{CMS:2017sdi,CMS:2017_lumi} in a correlated way.

\subsection{Theoretical uncertainties}
\label{sec:theory_unc}
The most important theoretical uncertainties arise from the modelling of the \ttbar+heavy flavour processes, namely \ttbb, \tttwob, \ttb~and \ttcc. Currently, it is not possible to isolate a pure~\ttbb~control region which would not contain a significant amount of~\ttHbb\xspace and thus this background cannot be determined directly from data. Although inclusive measurements of the~\ttbb~cross-section have been carried out at CMS~\cite{Sirunyan:2017snr} with $\sigma_{\ttbb}$ determined with a $\simeq 35\%$ relative accuracy, these analyses treat \ttHbb\xspace as an irreducible background and thus cannot directly be used to set the prior uncertainties in our analysis. On the other hand, NLO estimates for the inclusive cross-section of $\sigma_{\ttbb}$ have residual theoretical uncertainties on the level of $\simeq 20\%$~\cite{Bredenstein:2010rs}, with some important differences between alternative models. Therefore, we assign a conservative 50\% normalization uncertainty on all the \ttbar+heavy flavour processes, which is conservatively assumed to be uncorrelated across the aforementioned sub-processes. Effectively, this allows us to use the data in the control regions to determine the best fit values of the cross-sections for the \ttbar+heavy flavour processes in a consistent way with the extraction of the signal strength modifier.

The cross sections of all involved signal and background processes are known to at least NLO accuracy, with a 4\%~renormalization and factorization scale uncertainty and a 4\% PDF uncertainty on the gluon-gluon dominated production of~\ttbar~+jets. Shape uncertainties from PDF variations are found to be negligible and thus not considered further in the analysis. We use MC simulation to estimate the shape effect of the renormalization and factorization scale ($\mu_r$ and $\mu_f$) on the final discriminant shape by changing the nominal values by 0.5 (2.0) for the down (up) variation. This is achieved using the embedded ME-dependent weights in the MC simulation. The effect of these variations is illustrated on~\cref{fig:tth_scaleme} and is generally small on the final observables, but has a significant effect on the jet multiplicity and transverse momentum distributions.

\begin{figure}
\begin{centering}
\includegraphics[width=0.8\textwidth]{figures/tth/scaleME_unc.pdf}
\caption[The effect of $\mu_r$ and $\mu_f$ variations.]{The effect of the renormalization and factorization scale variations ($\mu_r$ and $\mu_f$) on the modelling of the jet multiplicity (left) in the SL $\geq4$jet, $\geq2$~b-tag inclusive region and the MEM discriminator (right) in the SL $\geq6$jet, $\geq4$~b-tag signal region. While the effect of the scale changes is normalized to be shape-changing in the inclusive region, it can introduce migrations between jet-tag bins among the final categories. These distributions are derived using \ttlf simulation.}
\label{fig:tth_scaleme}
\end{centering}
\end{figure}

For the parton shower uncertainties, in particular the effects of ISR and FSR, we have limited MC simulation samples that can only be used to determine the overall effect on normalization, whereas the shape distortions are generally consistent with no change. These background modelling uncertainties primarily affect the jet kinematics and thus the number of reconstructed jets in the final state. Therefore, we model these uncertainties through per-subprocess normalization factors that depends on the jet multiplicity, with the magnitude of the uncertainties generally between 5-15\%, as seen on~\cref{fig:tth_ttjets_modelling}. The overall normalization and shape effect of the most important shape changing uncertainties is shown on~\cref{fig:tth_uncertainties_effect}.

\begin{figure}
\begin{centering}
\includegraphics[width=1.0\textwidth]{figures/tth/CMS_ttjetsisr.pdf}
\caption[The \ttbar+jets ISR modelling uncertainties]{The \ttbar+jets ISR modelling uncertainties in terms of a scale factor that depends on jet multiplicity. We extract this scale factor by comparing the yield predicted by the variated MC simulation to the nominal, adding the statistical uncertainty on this prediction. Generally, these scale factors are symmetric around the nominal and have change between 85-115\%, with the most significant effects on the \ttlf~process}
\label{fig:tth_ttjets_modelling}
\end{centering}
\end{figure}

\begin{figure}
\begin{centering}
\includegraphics[width=1.0\textwidth]{figures/tth/uncs_sl_jge6_tge4.pdf}
\caption[The normalization and shape effect of uncertainties.]{The normalization and shape effect of the most important shape-changing uncertainties. On the top plot, we show the effect on the normalization in terms of the ratio to the nominal. On the bottom, we show the estimated effect on the shape of the template by computing the p-value of the $\chi^2$ test between the two distributions.}
\label{fig:tth_uncertainties_effect}
\end{centering}
\end{figure}

\subsection{Control regions}
We validate the simulation in the inclusive semileptonic and dileptonic control regions with at least 4 jets, out of which at least 2 must be b-tagged by comparing the simulated distributions of jet and lepton kinematic variables to data. The distributions in the semileptonic control region can be seen on~\cref{fig:tth_sl_control} and for the dileptonic on~\cref{fig:tth_dl_control}. In general, we see that both the inclusive yields and differential distributions for the kinematic variables are well-described within the systematic uncertainties. We observe a residual mismodeling in the jet multiplicity distribution, which we attribute to the \ttbar+jets MC tuning.

\begin{figure}
\begin{centering}
\subfloat{\includegraphics[width=0.5\textwidth]{figures/tth/sl_jge4_tge2/leps_0_pt.pdf}}
\subfloat{\includegraphics[width=0.5\textwidth]{figures/tth/sl_jge4_tge2/jetsByPt_0_pt.pdf}} \\

\subfloat{\includegraphics[width=0.5\textwidth]{figures/tth/sl_jge4_tge2/jetsByPt_0_btagCSV.pdf}}
\subfloat{\includegraphics[width=0.5\textwidth]{figures/tth/sl_jge4_tge2/numJets.pdf}} \\

\subfloat{\includegraphics[width=0.5\textwidth]{figures/tth/sl_jge4_tge2/nBCSVM.pdf}}
%\subfloat[The MEM discriminator distribution]{\includegraphics[width=0.4\textwidth]{figures/tth/sl_jge4_tge2.pdf}} \\
\caption[The modelling of the kinematic distributions in the semileptonic control region]{}
\label{fig:tth_sl_control}
\end{centering}
\end{figure}

\begin{figure}
\begin{centering}
\subfloat{\includegraphics[width=0.5\textwidth]{figures/tth/dl_jge4_tge2/leps_0_pt.pdf}}
\subfloat{\includegraphics[width=0.5\textwidth]{figures/tth/dl_jge4_tge2/leps_1_pt.pdf}} \\

\subfloat{\includegraphics[width=0.5\textwidth]{figures/tth/dl_jge4_tge2/jetsByPt_0_pt.pdf}}
\subfloat{\includegraphics[width=0.5\textwidth]{figures/tth/dl_jge4_tge2/jetsByPt_0_btagCSV.pdf}} \\

\subfloat{\includegraphics[width=0.5\textwidth]{figures/tth/dl_jge4_tge2/numJets.pdf}}
\subfloat{\includegraphics[width=0.5\textwidth]{figures/tth/dl_jge4_tge2/nBCSVM.pdf}} \\
\caption[The modelling of the kinematic distributions in the dileptonic control region]{}
\label{fig:tth_dl_control}
\end{centering}
\end{figure}


\subsection{Statistical method}
\label{sec:statistical_method}
In order to interpret the data, we use the same statistical framework as has been used for other Higgs boson searches in the CMS collaboration~\cite{Chatrchyan:2012xdj,Chatrchyan:2012tx,ATLAS:2011tau}. We wish to measure the signal strength modifier~$\mu = \sigma_{\ttH}/\sigma_{\ttH}^{\mathrm{SM}}$~and in the absence of an observed signal, exclude~$\mu \ge \mu^{CL}$~at a certain confidence level. The null hypothesis ($H_0$) is therefore the presence of a signal with a given~$\mu$, whereas the alternative hypothesis is no signal ($H_1, \mu = 0~$). Based on the data, we seek to exclude the null hypothesis above a certain~$\mu$.

The predicted distributions for both signal (denoted as~$s$) and background (denoted as~$b$) are subject to uncertainties introduced in~\cref{sec:systematic_unc} such that the expectations are functions of the nuisance parameters~$\theta$ through~$s(\theta)$~and~$b(\theta)$. The uncertainties are assumed to be either fully correlated or uncorrelated, as is more appropriate and conservative, which allows the likelihood function to be written in a factorized form.

To determine confidence intervals on the Higgs boson production cross section and thus quantify the presence or absence of a signal, we use the~$CL_s$~method~\cite{Junk:1999kv,Read:2002}, which defines the likelihood function~$\mathcal{L}(\mathrm{data} | \mu, \theta)$~as

\begin{align}
\label{eq:likelihood}
\mathcal{L}(\mathrm{data} | \mu, \theta) =&  \mathrm{Poisson}(\mathrm{data} | \mu \cdot s(\theta) + b(\theta)) \cdot p(\tilde{\theta} | \theta)\\
=& \prod_{i\in \mathrm{bins}} \frac{(\mu s_i + b_i)^{n_i}}{n_i!} \exp{[-(\mu s_i + b_i)]} \cdot p(\tilde{\theta} | \theta).
\end{align}
We have used Poisson probabilities to model the observation of~$n_i$~events in the bin~$i$~of a discretized distribution, given an expectation~$\mu s_i + b_i$. The distribution~$p(\tilde{\theta} | \theta)$~encodes the prior knowledge on the nuisance parameters, which have default values~$\tilde{\theta}$. This likelihood function can be computed both with observed data and with ``pseudo-data'', which is constructed from simulation under a specific hypothesis.

We use the test statistic~$\tilde{q}_\mu$, based on the profile likelihood ratio~\cite{Cowan:2010js}, to assess the compatibility of the data with either the \textit{background-only} or \textit{signal+background} hypotheses:

\begin{equation}
\tilde{q}_\mu = -2 \ln{\frac{\mathcal{L}(\mathrm{data} | \mu, \hat{\theta}_\mu)}{\mathcal{L}(\mathrm{data} | \hat{\mu}, \hat{\theta})}},\ 0 \le \hat{\mu} \le \mu.
\end{equation}
This test statistic is constructed such that it considers only models with~$\mu \ge 0$, furthermore it is constrained to be one sided by~$\hat{\mu} \le \mu$~such that data with~$\hat{\mu} > \mu$~are not used as part of the rejection region for the test on the upper limit of~$\mu$.

Here~$\hat{\theta}_\mu$~is the conditional maximum likelihood estimator of~$\theta$~given a fixed value~$\mu$, whereas~$\hat{\mu}$~and~$\hat{\theta}$~refer to the overall maximum likelihood estimators of both quantities. For a given signal strength modifier~$\mu$~that we test, we first find the observed value of~$\tilde{q}_\mu^{\mathrm{obs}}$~and the nuisance parameters~$\hat{\theta}_0$~(background hypothesis) and~$\hat{\theta}_\mu$~(signal hypothesis). Then, in order to compute the~$\mathrm{CL}_s(\mu)$, we compute the p-values of the signal and background hypotheses using

\begin{equation}
p_{\mu} = \int_{\tilde{q}_{\mu}{\mathrm{obs}}}^\infty f(\tilde{q}_{\mu} | \mu, \hat{\theta}_{\mu})\ \mathrm{d}\tilde{q}_\mu
\end{equation}
and

\begin{equation}
1 - p_b = \int_{\tilde{q}_{\mu}^{\mathrm{obs}}}^\infty f(\tilde{q}_{\mu} | 0, \hat{\theta}_0)\ \mathrm{d}\tilde{q}_{\mu}.
\end{equation}

The p-values are the probabilities of observing results as extreme or more given the underlying hypothesis and are derived from the probability densities of~$\tilde{q}_{\mu}$~under a given hypothesis:~$f(\tilde{q}_\mu | \mu, \hat{\theta}_\mu^{\mathrm{obs}})$. We find the 95\% confidence level on the upper limit of~$\mu$~by adjusting~$\mu$~until

\begin{equation}
\mathrm{CL}_s(\mu) = \frac{p_\mu}{1 - p_b} < 0.05.
\end{equation}
Equivalently, if~$\mathrm{CL}_s < \alpha$~at~a given $\mu$, then the Higgs boson is excluded at a production rate of~$\mu$~or higher with a confidence level~$1 - \alpha$.

In order to compute the upper limit on~$\mu$~given the observed data, we need the PDFs~$f(\tilde{q}_\mu | \mu, \hat{\theta}_\mu^{\mathrm{obs}})$, which can be derived using a Monte Carlo method by generating pseudo-data assuming the given signal strength~$\mu$~and fitting the observed data to evaluate the test statistic. As the MC procedure for generating the PDFs can be very time consuming, we use an approximate asymptotic distribution~\cite{Cowan:2010js} for the PDF~$\tilde{q}_\mu$, which results from the Wald approximation~\cite{wald1943tests}:

\begin{equation}
-2 \ln{\lambda(\mu)} = \frac{(\mu - \hat{\mu})^2}{\sigma^2}+ \mathcal{O}(1/\sqrt{N})
\end{equation}
where~$\sigma$~is the standard deviation of~$\hat{\mu}$~derived from the full covariance matrix of the likelihood function.

Using the asymptotic distribution for~$f(\tilde{q}_\mu | \mu, \hat{\theta}_\mu^{\mathrm{obs}})$, we find the upper limit for~$\mu$~at a confidence level of~$1 - \alpha$~to be

\begin{equation}
\mu = \hat{\mu} + \sigma \Phi^{-1}(1 - \alpha)
\end{equation}
where~$\Phi^{-1}$~is the inverse of the cumulative distribution of the Gaussian PDF. The standard deviation of~$\mu$~can be computed from the likelihood function~\cref{eq:likelihood} using the so-called Asimov data set, where the MC prediction is used for~$s_i$~and~$b_i$.

We will also quote the expected sensitivity of the measurement, which is derived from simulation, assuming~$\mu = 1$~and computing the median expected upper limit on~$\mu$~using the asymptotic formulae. 

\subsubsection{Uncertainties in the statistical model}

Our prior knowledge of the systematic uncertainties is encoded in~$p(\tilde{\theta} | \theta)$, where the nuisance parameters~$\theta$~are minimized in the profile likelihood method in a frequentist sense. We use~$\tilde{\theta}$~to represent our best estimate of the nuisance parameters, which can be
\begin{itemize}
\item Gaussian, used for shape variations,
\item log-normal used for nuisance parameters for which negative values are unphysical,
\item flat, in case we cannot assign a prior uncertainty.
\end{itemize}
In order to account for limited MC statistics, we create nuisance parameters for each process and bin in the template distributions, corresponding to the Poisson uncertainties from the limited number of simulation events.
%We use the Barlow-Beeston method in the fit to account for limited number of simulation events. In this method, the number of predicted events for a background component in a binned distribution is added as a nuisance parameter in the likelihood and minimized with the initial values arising from the observed Poisson counts using Newton's method~\cite{Barlow:1993dm}. 

\subsection{Analysis of the statistical model}
\label{sec:model_analysis}
In this section, we will study the expected sensitivity as predicted by the statistical model. For these studies, we use pseudo-data constructed according to the Asimov principle from MC simulation. First, we find that the likelihood function indeed has a minimum at $\mu=1$ ($\mu=0$) when evaluated on the corresponding Asimov dataset, as seen on~\cref{fig:tth_likelihood}.

\begin{figure}
\begin{centering}
\includegraphics[width = 0.5\textwidth]{figures/tth/r_scan.pdf}
\caption[The likelihood as a function of $\mu$]{The likelihood as a function of $\mu$ on the background only ($\mu=0$) Asimov dataset (blue) and the signal+background ($\mu=1$) Asimov dataset. We also see that the uncertainty in $\mu$ is reduced when removing the systematic uncertainties from the model.}
\label{fig:tth_likelihood}
\end{centering}
\end{figure}

Next, we study the effect of systematic uncertainties in the form of the shifts and constraints on the nuisance parameters after having been fit to pseudo-data derived from MC. We compute the pulls and constraints, defined as the central value and the width of the distribution $(\hat{\theta} - \theta) / \Delta\theta$ with respect to the pre-fit values, with the uncertainty $\Delta\theta$ determined around the minimum of the likelihood function. By fitting the signal+background and background-only models on the background-only dataset, we verify that both models result in equivalent constraints, as can be seen on~\cref{fig:tth_sldl_pulls}.

We further determine that in the fit to the $\mu=1$ Asimov dataset, some of the nuisances in the background-only model will be shifted with respect to their pre-fit values to compensate the mismatch between the model and the dataset. In particular, we observe a negative pull in the heavy flavour modelling of the b~discriminator and positive pulls for the \ttbb~normalization and \ttbar+jets ISR modelling, showing that these nuisance parameters are (anti)-correlated to the signal strength parameter $\mu$. When fitting the signal+background model on the $\mu=1$ dataset, we observe constraints at the level of $\mathrm{Var}[(\hat{\theta} - \theta_0)/\Delta \theta] \simeq 0.5$ for these nuisances, from which we surmise that the model has sensitivity for these nuisance parameters with respect to the prior uncertainties. This is expected, as the prior uncertainties on some of the nuisance parameters are assumed to be quite large to be conservative.

\begin{figure}
\begin{centering}
\subfloat{\includegraphics[width=0.32\textwidth]{figures/tth/pulls_group_sldl_sig0_r0_20_asimov.pdf}}
\subfloat{\includegraphics[width=0.32\textwidth]{figures/tth/pulls_group_sldl_sig0_r20_40_asimov.pdf}}
\subfloat{\includegraphics[width=0.32\textwidth]{figures/tth/pulls_group_sldl_sig0_r40_60_asimov.pdf}} \\

\subfloat{\includegraphics[width=0.32\textwidth]{figures/tth/pulls_group_sldl_sig1_r0_20_asimov.pdf}}
\subfloat{\includegraphics[width=0.32\textwidth]{figures/tth/pulls_group_sldl_sig1_r20_40_asimov.pdf}}
\subfloat{\includegraphics[width=0.32\textwidth]{figures/tth/pulls_group_sldl_sig1_r40_60_asimov.pdf}} \\
\caption[The pulls and constraints of the combined fit model with the Asimov datasets]{The pulls and constraints on the nuisances of the background-only model (blue) and the signal+background model (red) on the $\mu=0$ Asimov dataset (top row) and the $\mu=1$ Asimov dataset (bottom row). The nuisances are ordered by ascending constraint size (width of the pull distribution), shown as the error bar around the pull $\hat{\theta} / \theta$. For the $\mu=1$ Asimov dataset, we observe the strongest constraints at around $\hat{\sigma_{\theta}} \simeq 0.5$ for the CSV b~discriminator heavy flavour modelling (\texttt{CMS\_ttH\_CSVhf}), the \ttbb~normalization (\texttt{bgnorm\_ttbarPlusBBbar}), the \ttbar+jets FSR modelling (\texttt{CMS\_ttjetsfsr}) and the b~discriminator charm flavour modelling (\texttt{CMS\_ttH\_CSVcferr1}).}
\label{fig:tth_sldl_pulls}
\end{centering}
\end{figure}


\begin{figure}
\begin{centering}
\includegraphics[width = 0.7\textwidth]{figures/tth/nuis_scan.pdf}
\caption[The best-fit estimations of nuisance parameters as a function of the $\theta_{\ttbb}$ nuisance parameter]{The best-fit estimations of the signal strength parameter $\mu$ and other nuisance parameters as a function of the \ttbb\xspace normalization uncertainty ($\theta_{\ttbb}$) nuisance parameter. We see that the best-fit value of the signal strength parameter is anti-correlated to $\theta_{\ttbb}$, and other nuisance parameters have a non-trivial dependence on the value of $\theta_{\ttbb}$.}
\label{fig:tth_nuis_scan}
\end{centering}
\end{figure}

\section{Results}
\label{sec:tth_results}
After having validated the statistical model on simulation, we carry out the analysis on the observed data in the analysis categories by ``unblinding'' the data. We show the pre-fit and post-fit distributions in the final categories on~\cref{fig:tth_postfit1,fig:tth_postfit2,fig:tth_postfit3}. The pre-fit uncertainty is reduced by the fit and that the post-fit description of the data is generally very good, with a p-value for the goodness of fit test at the level of $p=1.0$. From this, we extract the best fit values of the signal strength parameter $\mu$ in the semileptonic and dileptonic categories separately and for both categories combined. We also find the upper limit on the signal strength at a 95\% confidence level, as shown in~\cref{eq:tth_bestfit} and on~\cref{fig:tth_combined}. The results can be visualized by sorting all the bins of the fitted templates according to the expected signal over background ratio, as shown on~\cref{fig:tth_sob}.

\begin{align}
\label{eq:tth_bestfit}
\hat{\mu} &= 1.0  ^{+0.3}_{-0.3}~\mathrm{stat} ^{+0.6}_{-0.6}~\mathrm{syst}\\
\mu^{95\%CL} &= 1.4~\mathrm{obs}~(1.4 ~\mathrm{exp}) \\
Z &= 1.3\sigma~\mathrm{obs}~(1.4 \sigma~\mathrm{exp})
\end{align}

\begin{figure}
\begin{centering}
\subfloat{\includegraphics[width=0.5\textwidth]{figures/tth/sl_j4_t3_prefit.pdf}}
\subfloat{\includegraphics[width=0.5\textwidth]{figures/tth/sl_j4_t3_postfit.pdf}} \\

\subfloat{\includegraphics[width=0.5\textwidth]{figures/tth/sl_j4_tge4_prefit.pdf}}
\subfloat{\includegraphics[width=0.5\textwidth]{figures/tth/sl_j4_tge4_postfit.pdf}} \\

\subfloat{\includegraphics[width=0.5\textwidth]{figures/tth/sl_j5_t3_prefit.pdf}}
\subfloat{\includegraphics[width=0.5\textwidth]{figures/tth/sl_j5_t3_postfit.pdf}}\\
\caption[The pre-fit and post-fit distributions in the semileptonic 4-jet and 5-jet categories]{The pre-fit and post-fit distributions in the semileptonic 4-jet and 5-jet categories}
\label{fig:tth_postfit1}
\end{centering}
\end{figure}

\begin{figure}
\begin{centering}
\subfloat{\includegraphics[width=0.5\textwidth]{figures/tth/sl_jge6_t3_prefit.pdf}}
\subfloat{\includegraphics[width=0.5\textwidth]{figures/tth/sl_jge6_t3_postfit.pdf}} \\

\subfloat{\includegraphics[width=0.5\textwidth]{figures/tth/sl_jge6_tge4_prefit.pdf}}
\subfloat{\includegraphics[width=0.5\textwidth]{figures/tth/sl_jge6_tge4_postfit.pdf}} \\
\caption[The pre-fit and post-fit distributions in the semileptonic $\geq6$-jet categories]{The pre-fit and post-fit distributions in the semileptonic $\geq6$-jet categories.}
\label{fig:tth_postfit2}
\end{centering}
\end{figure}

\begin{figure}
\begin{centering}
\subfloat{\includegraphics[width=0.5\textwidth]{figures/tth/dl_jge4_t3_prefit.pdf}}
\subfloat{\includegraphics[width=0.5\textwidth]{figures/tth/dl_jge4_t3_postfit.pdf}}\\

\subfloat{\includegraphics[width=0.5\textwidth]{figures/tth/dl_jge4_tge4_prefit.pdf}}
\subfloat{\includegraphics[width=0.5\textwidth]{figures/tth/dl_jge4_tge4_postfit.pdf}}\\
\caption[The pre-fit and post-fit distributions in the dileptonic categories]{The pre-fit and post-fit discriminator distributions in the dileptonic categories.}
\label{fig:tth_postfit3}
\end{centering}
\end{figure}

\begin{figure}
\begin{centering}
\includegraphics[width = 1.0\textwidth]{figures/tth/nuis_corr.pdf}
\caption[The post-fit correlation between the signal strength parameter and the nuisance parameters]{The post-fit correlation coefficient between the signal strength parameter $\mu$ and the nuisance parameters.}
\label{fig:tth_corr}
\end{centering}
\end{figure}

\begin{figure}
\begin{centering}
\subfloat{\includegraphics[width = 0.5\textwidth]{figures/tth/bestfit.pdf}} 
\subfloat{\includegraphics[width = 0.5\textwidth]{figures/tth/limits_comb.pdf}} 
\caption[The best-fit value and the 95\% upper limits on the signal strength parameter]{The best-fit value and the 95\% upper limits on the signal strength parameter}
\label{fig:tth_combined}
\end{centering}
\end{figure}

\begin{figure}
\begin{centering}
\includegraphics[width = 0.8\textwidth]{figures/tth/sob.pdf}
\caption[The final analysis bins, arranged by $S/B$.]{The best-fit signal and background distributions in the final analysis bins, arranged by $S/B$, along with the data.}
\label{fig:tth_sob}
\end{centering}
\end{figure}

\subsubsection{Systematic uncertainties}
After fitting the model to data, we determine the impacts of the major uncertainties on the signal strength parameter, with the results shown in~\cref{tab:systematic_uncertainties_posterior}. These are determined by freezing a particular set of nuisance parameters and determining the change in the post-fit uncertainty on $\mu$, under the assumption that the uncertainties are uncorrelated. As this assumption holds only approximately, as can be seen from the correlation matrix on~\cref{fig:tth_corr}, the individual uncertainty components do not sum up precisely to the total uncertainty in quadrature.

Overall, we see that this analysis is dominated by systematic uncertainties in the theoretical modelling of the background processes. In particular, the considerable normalization uncertainties assigned to the various \ttbar+heavy flavour processes, specifically the \ttbb~process, have a significant effect on the analysis. This is followed by the scale uncertainties resulting in distortions to the final discriminant shapes. The detailed theoretical uncertainties arising from the parton shower (ISR, FSR) and the specifics of the MC tune are smaller, but not negligible.

Out of the experimental uncertainties, the major sources of uncertainty are MC simulation statistics, which remain a challenge in the high-multiplicity final state, followed by the detailed modelling of the experimental observables related to b-tagging and jet energy corrections. In general, all these sources of uncertainty have considerable effects on the predicted number of events and the distributions of the final discriminators in the analysis categories, such that the final nuisance parameters have non-negligible correlations.

\begin{figure}
\begin{centering}
\includegraphics[width = 0.8\textwidth]{figures/tth/impacts.pdf}
\caption[The impact of nuisance parameters on the signal strength.]{The impact of nuisance parameters on the signal strength. The impacts are derived by shifting the nuisance parameter under question by one standard deviation up or down around the post-fit value $\hat{theta}$ and computing the change in the best-fit signal strength value.}
\label{fig:tth_impacts}
\end{centering}
\end{figure}

\begin{table}[h!]
\begin{center}
\begin{tabular}{c|ccc}
\hline
uncertainty & $\Delta\mu_-$ & $\Delta\mu_+$ \\
\hline
\ttbar+jets (norm.) & -0.24 & +0.16\\
MC statistics & -0.21 & +0.21\\
$\mu_R$, $\mu_F$ scale & -0.11 & +0.17\\
b-tagging & -0.11 & +0.13\\
ISR, FSR & -0.04 & +0.11\\
JEC & -0.07 & +0.07\\
pdf (norm.) & -0.01 & +0.06\\
MC tune, $h_{\mathrm{damp}}$ & -0.08 & +0.12\\
lumi, pileup, lepton & -0.07 & +0.08\\
\hline
experimental & -0.25 & +0.26\\
theory & -0.45 & +0.45\\
\hline
systematic & -0.66 & +0.69\\
statistical & -0.31 & +0.31\\
\hline
\hline
\end{tabular}
\caption[The post-fit uncertainties in the~\ttHbb\xspace analysis]{The post-fit uncertainties on the signal strength parameter $\mu$ in the~\ttHbb\xspace analysis. The uncertainties are estimated by freezing sets of nuisance parameters in the fit and evaluating the change in post-fit uncertainty around the best-fit value.}
\label{tab:systematic_uncertainties_posterior}
\end{center}
\end{table}

\section{Summary}
\label{sec:tth_summary}
We have presented a search for the \ttH\xspace process in the data collected by the CMS experiment during the 2016 run period, in the channel where the Higgs boson decays to b~quarks and at least one of the top quarks decays leptonically. The observation of this process, where the Higgs boson is produced in association with top quarks, would make it possible to directly study the process of mass generation for up-type quarks. The analysis is challenging due to the presence of a significant background arising from the QCD production of \ttbar+jets. Specifically, the \ttbb\xspace process, where two additional b~quarks are produced in the final state in association with the top quark pair, is irreducible with respect to the particles in the final state, as we we expect 4-6 jets, out of which 4 arise from b~quarks from both this process and the \ttHbb\xspace signal. Furthermore, the analysis is complicated by the presence of a combinatorial self-background, as we cannot directly reconstruct the Higgs boson candidate invariant mass peak due to the presence of several additional b~quark candidates arising from the top decay.

We have shown that it is possible construct a discriminator based on the direct computation of scattering amplitudes on the kinematic properties of the observed jets and leptons in the event, without relying on large amounts of MC simulation for the \ttbb process, which is affected by considerable theoretical uncertainties. We have been able to establish an upper limit on the signal strength modifier at the level of $\mu < 1.4$ at a confidence level of 95\%, with $1.4$ expected under the SM hypothesis. The best fit signal strength value is $\mu = 1.0  ^{+0.3}_{-0.3}~\mathrm{stat} ^{+0.6}_{-0.6}~\mathrm{syst}$, with the total combined uncertainty being $0.7$. The largest components in the final uncertainty are of systematic origin, arising from the modelling of the \ttbar+heavy flavour background and the overall uncertainties in the \ttbar+jets production. Additionally, we find simulation statistics and experimental uncertainties in the detailed calibration of the b~discriminator shape to have a sub-leading but significant effect on the final measurement. This means that in addition to an improved theoretical treatment of the \ttbar+jets background, improving the MC simulation in terms of a more efficient use of the generated events and an improved modelling of the quantities related to b-tagging can improve the accuracy of this analysis. Both of these problems are well-suited to be tackled with the matrix element method discriminator, which does not rely singificantly on b-tagging or extensive simulation statistics.  
