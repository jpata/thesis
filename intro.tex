\chapter*{Introduction}
\addcontentsline{toc}{chapter}{Introduction}
\markboth{Introduction}{}

The aim of particle physics is to describe and understand the fundamental laws of nature. The Universe has been found to consist of particles, the quarks and leptons, and the interactions between these particles, mediated by weak, strong and electromagnetic forces. The standard model (SM) of particle physics is the achievement of more than a century of experimental and theoretical work by the physics community, starting from the discovery of the electron by J. J. Thomson in the end of the 19th century and the atomic nucleus by E. Rutherford in the early 20th century. Crucially, it is a mathematical theory that can be used to describe high-energy processes in physics with a high degree of accuracy, as has been confirmed by collider experiments. The discovery of the Higgs boson in 2012 confirmed the basic mechanism of mass generation through electroweak symmetry breaking, but many open questions remain.

Several observed phenomena do not find an explanation, or at least a sufficient one, in present-day particle physics. In particular, the existence of non-luminous gravitating matter or dark matter (DM) is necessary to explain features on the astrophysical and cosmological scales, such as the rotation curves of galaxies and the energy spectrum of the cosmic microwave background radiation. It is plausible that this gravitational effect results from weakly interacting particles that that are distinct from quarks or leptons. It is possible these DM particles could be detected (I) in collider searches, (II) by directly observing the nuclear recoil from the DM halo in our galaxy due to weak interactions, or (III) indirectly through the detection of cosmic rays resulting from DM annihilations.

Furthermore, the non-zero neutrino masses and mixings, necessary to explain the observed neutrino flavour oscillations, currently do not have an explanation in the SM. It is possible that neutrinos, which have masses between six to ten orders of magnitude below other fermions, acquire mass by a different mechanism than Yukawa interactions with the Higgs field. Thus, research in neutrino physics is expected to shed light on the nature, the mass generation mechanism and hierarchy and the mixing properties of the neutrinos. There is considerable asymmetry between matter and anti-matter in the Universe, but the sources of charge-parity (CP) violation that would allow this to happen are not sufficient in the SM. The experimental study of the flavour sector of high-energy physics complements direct searches for new physics at the energy frontier.

The Higgs field is the first observed fundamental scalar field, therefore, it is interesting to study the properties of the observed Higgs boson more deeply. It is possible that the Higgs boson is a composite particle, a hypothesis which can be tested by precisely measuring the couplings of the Higgs boson to SM particles.

There are other aspects of the SM that cannot yet be explained satisfactorily. In particular, fermions are found to fall into three generations, but there seems to be no underlying reason for this. Furthermore, many of the parameters of the SM, for example the masses and mixing properties of the fermions, cannot be deduced from theory alone and have to be determined experimentally. All of the aforementioned questions motivate a further study of the physics of high-energy processes, with the hope of arriving at a more complete mathematical understanding of the Universe.
The Large Hadron Collider (LHC) experiments, which in the last years have collected a few percent of the data foreseen over their lifetimes and have experimentally confirmed the foundations of the SM, are essential for progress on the aforementioned questions.

In this thesis, we focus on improving our understanding of the Higgs sector by developing a measurement for the Higgs production cross-section in a rare production mode, where the Higgs boson is produced in association with a top quark pair. The experimental confirmation of the \ttH~process would be a direct confirmation of the mechanism for mass generation for the most massive known quark. It is expected that using Run 2 data, the presence of this production mode can be determined with a significance exceeding five Gaussian standard deviations ($\sigma$) and thus the Yukawa coupling of the top quark can be measured with a precision of about $\simeq10\%$. This requires effort on the experimental side in reducing the effect of experimental uncertainties and of the background contribution arising from QCD production of top quark pairs, which is the overall focus of this thesis. We have developed a data analysis method based on the direct computation of matrix elements on observed events that does not rely on the simulation and subsequent fitting of millions of complex multi-jet events using Monte Carlo. This method is applied in the \ttHbb~search using $35.9~\ifb$~of data collected during 2016 by the CMS experiment to extract upper limits on the \ttH~production cross-section.

The accurate reconstruction and identification of jets is essential for this search. We have improved upon the algorithms used for the identification of jets from the hadronisation of bottom quarks at CMS by developing a classifier based on machine learning that combines information from various subsystems of the detector. This improved algorithm was used during data-taking at CMS during the 2016 data taking period.

The LHC experiments will face unprecedented data rates in the coming decades during the high-luminosity LHC (HL-LHC) project, where the total amount of collected data will increase by two orders of magnitude. This presents an opportunity for physics, but also a challenge in terms of the reduction, analysis and storage of these data. Recent advances in the field of machine learning have made it possible to use algorithms that directly learn from and adapt to data, instead of being constructed by human experts. Using these techniques based on mathematical optimisation, human-level performance has been achieved or exceeded in many areas where an algorithmic solution was previously thought to be many decades away, such as vision or the game of Go. During an internship at the private company Lingvist we applied these machine learning methods to model language learning behaviour in humans, where theoretical models are less predictive than in physics. The use of such data-driven techniques shows great promise in fields where the underlying model is not known, but can also benefit physics in cases where theoretical models are not yet predictive enough, such as the reconstruction and identification of complex signals spanning the detector.

This thesis is structured as follows. In ~\cref{sec:theory}, we introduce the Standard Model of particle physics and the theoretical background for Higgs physics. We discuss the experimental setup of the LHC machine and the CMS experiment in ~\cref{sec:experiment}. The method used at CMS for identifying jets from bottom quarks is introduced in ~\cref{sec:btagging}, where we also discuss our contribution to the CMS state of the art. We introduce the matrix element method used for the \ttHbb search in ~\cref{sec:mem} and discuss the implementation, improvements and validation studies that we carried out. This method is applied in a search for \ttHbb~using CMS data, which is described in ~\cref{sec:tth}. We describe the data and simulation samples, the reconstruction of the signal, the systematic uncertainties affecting the measurement and the procedure used to extract the signal strength parameter and the upper limit on \ttH~production. We also compare our results to the latest analysis from the ATLAS collaboration. Finally, in ~\cref{sec:lingvist} we discuss our work on modelling language learning using data-driven machine learning techniques. We conclude with a summary and outlook in ~\cref{sec:outlook}.

\chapter*{Acknowledgements}
\addcontentsline{toc}{chapter}{Acknowledgements}
\markboth{Acknowledgements}{}
I would like to acknowledge some of the people from whom I've benefited the most during my PhD studies. First and foremost, my deep appreciation goes to G\"unther Dissertori for accepting me as a student and for his advice and support during these years. Secondly, I would like to thank Lorenzo Bianchini and Gregor Kasieczka, who were very generous with their time and from whom I was able to learn a lot during this project. I would also like to thank Nigel Glover and the rest of the people who created the HiggsTools Initial Training Network, which made it possible for me to undertake PhD studies at ETH Z\"urich. I'm very grateful to Joe Incandela, whose generosity via the CMS Fundamental Physics Scholarship supported my first year at CERN. I learned a lot from all my colleagues at the Institute of Particle Physics and Astrophysics at ETH Z\"urich, in particular Pasquale Musella and Malte Backhaus, whose comments were essential for this thesis. I'm thankful to my Estonian HEP colleagues at NICPB, Tallinn, thanks to whom I was able to learn about experimental particle physics at CERN during my undergraduate studies. It was very interesting to work with the Lingvist team in Tallinn during my internship and I'm thankful for their kind hospitality. Finally, I would like to thank my parents and my family for nurturing and encouraging my interest in science.