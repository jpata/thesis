\chapter{Conclusions and outlook}
\label{sec:outlook}
The SM is a successful and predictive theory of high-energy processes. Nevertheless, it is clearly not a complete theory of the physical Universe. Among the open questions, the nature of the electroweak symmetry breaking mechanism needs to be clarified experimentally. Deviations between the couplings of the recently-discovered Higgs boson and SM particles can arise in BSM scenarios and thus must be tested experimentally. The Large Hadron Collider and the corresponding general purpose detector experiments, CMS and ATLAS, are in a unique position to clarify the nature of this fundamental scalar. The couplings of the Higgs boson to vector bosons have been determined in Run 1 of the LHC to a relative precision of about $\simeq10\%$. In Run 2, thanks to the increased centre-of-mass energy and luminosity, the couplings to the fermion sector can be explored in more detail.

The focus of this thesis was the search for the top quark pair associated Higgs production mode (\ttH) in the channels where at least one of the top quarks decays leptonically and the Higgs boson decays to bottom quarks. This is a sensitive channel which contributes to the overall \ttH~cross-section measurement and thus the determination of the top quark Yukawa coupling. However, the analysis is challenging due to the presence of an irreducible background arising from the QCD production of top quark pairs and the complex multi-jet final state with a combinatorial ambiguity between the jets. Multivariate analysis techniques that exploit the differences in the dynamics of the signal and background processes are necessary to achieve sufficient sensitivity. Traditionally, LHC analyses have employed machine learning techniques, where large amounts of MC simulation are used to determine a signal-to-background classifier. This approach works well and has been essential for many of the results of the LHC experiments. However, for final states with a large number of jets, it is challenging to generate sufficient MC simulation events that pass the experimental cuts in the most sensitive regions of the phase space, making it difficult to optimise the machine learning algorithms. Such predictions also have significant theoretical uncertainties arising from the challenges of matching the hard process to the parton shower and the multiple momentum scales present in the hard process.

We have investigated an alternative approach for analysis, where the signal and background processes are distinguished by making use of the matrix elements directly at the event observable level. We made significant contributions to the matrix element method as applied to \ttHbb~in CMS and this approach is now standard within the collaboration for this analysis. We have also carried out a complete search for \ttHbb~in the leptonic channels in the 2016 data collected by CMS using the matrix element method. We have not observed any significant excess and have thus established an upper limit on the signal strength factor $\mu = \sigma_{\ttH}/\sigma_{\ttH}^{\mathrm{SM}}$~at 1.52 (1.57 expected) at a confidence level of 95\%. The upcoming years and the data collected by the High-Luminosity LHC project promise to be particularly interesting for \ttH, with a discovery of this process being within reach in the next years. On the other hand, theoretical and experimental uncertainties are significant in this analysis. The former can be reduced by using more accurate theoretical models, which are increasingly becoming available. There, the matrix element method can be a useful way to incorporate latest predictions into the analysis. For the latter, the analysis techniques, in particular the reconstruction of jets and the identification of jets from bottom quarks need to be refined.

For object reconstruction and identification, machine learning presents a useful way for combining signals across various channels. We have investigated a method for b~jet identification where several independently optimised b~discriminators based on different vertexing algorithms, track properties and presence of leptonic decays of hadrons are combined into a multivariate super-discriminator between jets arising from bottom quarks, light quarks and charm flavoured quarks. This approach presented an improvement over the state-of-the art at CMS and was deployed during the 2016 data taking and used in several physics analyses. The success of this discriminator relied partly on benefiting from the developments made in the field of data science outside of high-energy physics. Currently, such discriminators rely fully on simulation of the hard scattering, the subsequent showering and hadronisation and the detector simulation and thus are assigned significant uncertainties based on data-to-simulation corrections. An interesting direction for research would be to reduce the sensitivity of such discriminators by making use of the theoretical and experimental uncertainties during the optimisation phase. The direct use of collider data together with the appropriate simulation in an semi-supervised learning environment may be possible to reduce model dependence.

Although the SM of particle physics predicts many observable quantities accurately from a relatively small set of principles and parameters, there are features that are less understood due to the presence of non-perturbative physics, for example hadronisation or the precise parton content of protons in high-energy processes. In such cases, modelling the observed data in a phenomenological way and performing global fits presents a way to make testable predictions, as has been evidenced by the usefulness of effective hadronisation models and PDF fits for LHC experiments. During an internship at the private company Lingvist Technologies, we investigated and developed such methods in the field of human cognition and language learning. We proposed a data-driven model based on deep learning that predicted the performance of learners in a vocabulary test. Using such predictions, it is possible to optimise the learning process and make it personal on a large scale. Our proposed model improved significantly over the existing approach employed by the learning environment offered by the company. Such data-driven methods may also be useful in physics for problems where our understanding is not yet sufficient to have accurate predictions based on well-understood theory. However, it remains an open question how to best develop these phenomenological models such that they take into account the physical principles that have been well established by experiments.

In conclusion, the successful operation of the LHC and the detectors has opened up a new program in fundamental physics, making it possible to experimentally study the properties of the recently discovered Higgs boson, the only known fundamental scalar. 